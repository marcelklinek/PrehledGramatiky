\documentclass[10pt,a4paper]{article}
\usepackage{geometry}
\usepackage{cmap}
\usepackage[T1, IL2]{fontenc}
\usepackage{lmodern}
\usepackage[czech]{babel}
\usepackage[utf8]{inputenc}
\usepackage{indentfirst}
\usepackage{csquotes}
\usepackage{makecell}
\usepackage{booktabs}
\usepackage{microtype}

\geometry{
    paper=a4paper, % Change to letterpaper for US letter
    top=1cm, % Top margin
    bottom=1cm, % Bottom margin
    left=0.5cm, % Left margin
    right=0.5cm, % Right margin
    %showframe, % Uncomment to show how the type block is set on the page
}

\begin{document}
 \thispagestyle{empty} 
\begin{table}[!h]
\caption{Vzory mužského rodu}
\smallskip
\begin{tabular}{llll}
\begin{tabular}{lll}
\toprule
\multicolumn{3}{c}{\textbf{PÁN}} \\
\midrule\relax
1. & pán & páni, -ové\\
& & (občané) \\
\midrule\relax
2. & pána & pánů  \\
\midrule\relax
3. & pánu, -ovi & pánům  \\
\midrule\relax
4. & pána & pány \\
\midrule\relax
5. & pane! & páni! -ové! \\
& (hochu!) & (občané!) \\
\midrule\relax
6. & pánu, -ovi & pánech \\
& & (hoších) \\
\midrule\relax
7. & pánem & pány \\
\bottomrule
\end{tabular}
&
\begin{tabular}{lll}
\toprule
\multicolumn{3}{c}{\textbf{MUŽ}} \\
\midrule\relax
1. & muž & muži, -ové \\
& & (obyvatelé) \\
\midrule\relax
2. & muže & mužů  \\
\midrule\relax
3. & muži, -ovi & mužům  \\
\midrule\relax
4. & muže & muže \\
\midrule\relax
5. & muži! & muži! -ové! \\
& (otče!) & (obyvatelé!) \\
\midrule\relax
6. & muži, -ovi & mužích \\
& & \\
\midrule\relax
7. & mužem & muži \\
\bottomrule
\end{tabular}
&
\begin{tabular}{lll}
\toprule
\multicolumn{3}{c}{\textbf{HRAD}} \\
\midrule\relax
1. & hrad & hrady \\
& & \\
\midrule\relax
2. & hradu (lesa) & hradů  \\
\midrule\relax
3. & hradu & hradům  \\
\midrule\relax
4. & hrad & hrady \\
\midrule\relax
5. & hrade! & hrady! \\
& (kalichu!) & \\
\midrule\relax
6. & hradě, -u & hradech \\
& & (zámcích) \\
\midrule\relax
7. & hradem & hrady \\
\bottomrule
\end{tabular}
&
\begin{tabular}{lll}
\toprule
\multicolumn{3}{c}{\textbf{ČAJ}} \\
\midrule\relax
1. & čaj & čaje \\
& & \\
\midrule\relax
2. & čaje & čajů  \\
\midrule\relax
3. & čaji & čajům  \\
\midrule\relax
4. & čaj & čaje \\
\midrule\relax
5. & čaji! & čaje! \\
& & \\
\midrule\relax
6. & čaji & čajích \\
& & \\
\midrule\relax
7. & čajem & čaji \\
\bottomrule
\end{tabular}

\\
\\
\begin{tabular}{lll}
\toprule
\multicolumn{3}{c}{\textbf{PŘEDSEDA}} \\
\midrule\relax
1. & předseda & předsedové \\
& & (husité) \\
\midrule\relax
2. & předsedy & předsedů  \\
\midrule\relax
3. & předsedovi & předsedům  \\
\midrule\relax
4. & předsedu & předsedy \\
\midrule\relax
5. & předsedo! & předsedové! \\
& & (husité!) \\
\midrule\relax
6. & předsedovi & předsedech \\
& & \\
\midrule\relax
7. & předsedou & předsedy \\
\bottomrule
\end{tabular}
&
\begin{tabular}{lll}
\toprule
\multicolumn{3}{c}{\textbf{SOUDCE}} \\
\midrule\relax
1. & soudce & soudci \\
& & (soudcové) \\
\midrule\relax
2. & soudce & soudců  \\
\midrule\relax
3. & soudci, -ovi & soudcům  \\
\midrule\relax
4. & soudce & soudci \\
\midrule\relax
5. & soudce! & soudci! \\
& & (soudcové!) \\
\midrule\relax
6. & soudci, -ovi & soudcích \\
& & \\
\midrule\relax
7. & soudcem & soudci \\
\bottomrule
\end{tabular}
&
\begin{minipage}{5cm}
\textbf{Jak určit vzor?}
\begin{enumerate}
\setlength\itemsep{0em}
\item Určit rod. Použijeme ukazovací zájmeno (ten, ta, to)
\item Určit vzor podle \textbf{1. pádu Sg}. Křeslo: to křeslo — rod s., koncovka -o — vzor \textit{město}. Liška: ta liška — rod ž., koncovka -a — vzor \textit{žena}.
\item Občas je třeba \textbf{upřesnit podle 2. pádu.} Vejce: to vejce, rod s., bez vejce jako bez moře — vzor \textit{moře}.
\end{enumerate}
\end{minipage}

\end{tabular}
\end{table}

\begin{table}[!h]
\caption{Vzory ženského rodu}
\smallskip
\begin{tabular}{llll}
\begin{tabular}{lll}
\toprule
\multicolumn{3}{c}{\textbf{ŽENA}} \\
\midrule\relax
1. & žena & ženy\\
\midrule\relax
2. & ženy & žen  \\
\midrule\relax
3. & ženě & ženám  \\
\midrule\relax
4. & ženu & ženy \\
\midrule\relax
5. & ženo! & ženy! \\
\midrule\relax
6. & ženě & ženách \\
\midrule\relax
7. & ženou & ženami \\
\bottomrule
\end{tabular}
&
\begin{tabular}{lll}
\toprule
\multicolumn{3}{c}{\textbf{RŮŽE}} \\
\midrule\relax
1. & růže & růže\\
\midrule\relax
2. & růže & růží (ulic)  \\
\midrule\relax
3. & růži & růžím  \\
\midrule\relax
4. & růži & růže \\
\midrule\relax
5. & růže! & růže! \\
\midrule\relax
6. & růži & růžích \\
\midrule\relax
7. & růží & růžemi \\
\bottomrule
\end{tabular}
&
\begin{tabular}{lll}
\toprule
\multicolumn{3}{c}{\textbf{PÍSEŇ}} \\
\midrule\relax
1. & píseň & písně \\
\midrule\relax
2. & písně & písní  \\
\midrule\relax
3. & písni & písním  \\
\midrule\relax
4. & píseň & písně \\
\midrule\relax
5. & písni! & písně! \\
\midrule\relax
6. & písni & písních \\
\midrule\relax
7. & písní & písněmi \\
\bottomrule
\end{tabular}
&
\begin{tabular}{lll}
\toprule
\multicolumn{3}{c}{\textbf{KOST}} \\
\midrule\relax
1. & kost & kosti \\
\midrule\relax
2. & kosti & kostí  \\
\midrule\relax
3. & kosti & kostem  \\
\midrule\relax
4. & kost & kosti \\
\midrule\relax
5. & kosti! & kosti! \\
\midrule\relax
6. & kosti & kostech \\
\midrule\relax
7. & kostí & kostmi \\
\bottomrule
\end{tabular}
\end{tabular}
\end{table}


\begin{table}[!h]
\caption{Vzory středního rodu}
\smallskip
\begin{tabular}{llll}
\begin{tabular}{lll}
\toprule
\multicolumn{3}{c}{\textbf{MĚSTO}} \\
\midrule\relax
1. & město & města \\
\midrule\relax
2. & města & měst \\
\midrule\relax
3. & městu & městům  \\
\midrule\relax
4. & město & města \\
\midrule\relax
5. & město! & města! \\
\midrule\relax
6. & městě, -u & městech \\
& (jablku) & (jablkách, \\
& & jablcích) \\
\midrule\relax
7. & městem & městy \\
\bottomrule
\end{tabular}
&
\begin{tabular}{lll}
\toprule
\multicolumn{3}{c}{\textbf{MOŘE}} \\
\midrule\relax
1. & moře & moře\\
\midrule\relax
2. & moře & moří \\
\midrule\relax
3. & moři & mořím \\
\midrule\relax
4. & moře & moře \\
\midrule\relax
5. & moře! & moře! \\
\midrule\relax
6. & moři & mořích \\
& & \\
& & \\
\midrule\relax
7. & mořem & moři \\
\bottomrule
\end{tabular}
&
\begin{tabular}{lll}
\toprule
\multicolumn{3}{c}{\textbf{KUŘE}} \\
\midrule\relax
1. & kuře & kuřata \\
\midrule\relax
2. & kuřete & kuřat \\
\midrule\relax
3. & kuřeti & kuřatům \\
\midrule\relax
4. & kuře & kuřata \\
\midrule\relax
5. & kuře! & kuřata! \\
\midrule\relax
6. & kuřeti & kuřatech \\
& & \\
& & \\
\midrule\relax
7. & kuřetem & kuřaty \\
\bottomrule
\end{tabular}
&
\begin{tabular}{lll}
\toprule
\multicolumn{3}{c}{\textbf{STAVENÍ}} \\
\midrule\relax
1. & stavení & stavení \\
\midrule\relax
2. & stavení & stavení  \\
\midrule\relax
3. & stavení & stavením  \\
\midrule\relax
4. & stavení & stavení \\
\midrule\relax
5. & stavení! & stavení! \\
\midrule\relax
6. & stavení & staveních \\
& & \\
& & \\
\midrule\relax
7. & stavením & staveními \\
\bottomrule
\end{tabular}
\end{tabular}
\end{table}

\end{document}